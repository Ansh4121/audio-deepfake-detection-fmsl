\chapter{Conclusion and Future Work}

\section{Holistic Summary of Research Journey}

\subsection{Complete Research Narrative}
This thesis presented a systematic approach to improving audio deepfake detection through geometric analysis and targeted solution development. Beginning with the observation that existing models struggle to generalize to unseen attack types \cite{muller2022generalize}, we conducted a comprehensive analysis of eight progressive model architectures to identify the core limitation: insufficient angular separation in the learned feature space.

Our proposed solution, Feature Matching Self-Supervised Learning (FMSL), directly addresses this geometric bottleneck by combining L2 normalization, angular margin learning \cite{arcface}, and prototype-based spoof modeling. The approach is inspired by successful techniques in face recognition but adapted to the unique challenges of audio deepfake detection.

\subsection{Key Achievements}
The research contributions of this thesis include:
\begin{enumerate}
\item \textbf{Systematic limitation identification}: Through analysis of Maze 1--8 architectures, we identified the geometric bottleneck as the primary factor limiting generalization
\item \textbf{Novel FMSL framework}: A plug-and-play module that improves any baseline architecture without modifying the feature extractor
\item \textbf{Consistent improvements}: FMSL achieves improvements across all tested architectures, with Maze 5 + FMSL achieving optimal performance
\item \textbf{Standardized evaluation}: A reproducible framework ensuring fair comparison across model variants
\end{enumerate}

\section{Driving the Point Home: Comprehensive Results}

\subsection{Summary Visualizations}
The experimental results (Chapter 6) demonstrate that FMSL consistently improves detection performance across all Maze architectures. Key visualizations include:
\begin{itemize}
\item Performance comparison charts showing EER improvements
\item Score distribution analysis revealing improved class separation
\item Confusion matrix analysis demonstrating reduced false acceptance rates
\end{itemize}

All figures are available in the repository at \texttt{WUT-Thesis/img/} and can be regenerated using the scripts in \texttt{Thesis/02\_Evaluation\_Scripts/}.

\subsection{Key Performance Metrics}
On the ASVspoof 2019 LA evaluation set \cite{asvspoof2019evaluation}:
\begin{itemize}
\item Maze 5 + FMSL achieves the optimal balance of performance and complexity
\item FMSL provides consistent improvements regardless of baseline architecture
\item The standardized protocol ensures fair, reproducible comparisons
\end{itemize}

\section{Future Research Directions}

\subsection{Potential Extensions}
Future work could explore:
\begin{enumerate}
\item \textbf{Cross-dataset evaluation}: Testing FMSL on ASVspoof 2021 and other benchmarks \cite{huang2025}
\item \textbf{Adaptive prototypes}: Dynamically adjusting the number of prototypes based on attack diversity
\item \textbf{Multi-modal fusion}: Combining audio features with linguistic or visual cues for enhanced detection
\item \textbf{Real-time deployment}: Optimizing FMSL for low-latency inference in production systems
\end{enumerate}

\subsection{Open Questions}
Several open questions remain for the research community:
\begin{itemize}
\item How can detection systems maintain robustness against continuously evolving synthesis technologies?
\item What is the theoretical limit of generalization for learned countermeasures?
\item Can self-supervised pre-training further improve FMSL performance?
\end{itemize}

The code and materials for reproducing all experiments are available at:\\ \url{https://github.com/Ansh4121/audio-deepfake-detection-fmsl}

