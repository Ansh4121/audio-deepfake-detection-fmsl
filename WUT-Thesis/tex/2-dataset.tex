\chapter{Dataset and Preprocessing}

\section{ASVspoof2019 Dataset Description}

\subsection{Dataset Characteristics}
The ASVspoof 2019 Logical Access (LA) dataset \cite{asvspoof2019} serves as the primary benchmark for this research. It contains bonafide speech recordings and spoofed audio generated using 19 different text-to-speech (TTS) and voice conversion (VC) systems. The dataset is specifically designed to evaluate countermeasure systems against logical access attacks.

Key statistics:
\begin{itemize}
\item \textbf{Training set}: 25,380 bonafide + 25,380 spoofed utterances (6 known attack types: A01--A06)
\item \textbf{Development set}: 2,548 bonafide + 22,296 spoofed utterances
\item \textbf{Evaluation set}: 7,355 bonafide + 63,882 spoofed utterances (13 unknown attack types: A07--A19)
\item \textbf{Sample rate}: 16 kHz
\item \textbf{Format}: FLAC audio files
\end{itemize}

\subsection{Evaluation Metrics}
Following the ASVspoof 2019 evaluation protocol \cite{asvspoof2019evaluation}, we use:
\begin{itemize}
\item \textbf{Equal Error Rate (EER)}: The operating point where false acceptance rate equals false rejection rate---the primary metric for comparing systems.
\item \textbf{minimum tandem Detection Cost Function (min-tDCF)}: A cost-weighted metric that considers both countermeasure and automatic speaker verification performance.
\item \textbf{Accuracy}: Classification accuracy on the evaluation set.
\end{itemize}

\section{Data Processing Pipeline}

\subsection{Feature Extraction Process}
The preprocessing pipeline follows standardized protocols:
\begin{enumerate}
\item \textbf{Audio loading}: Load FLAC files at native 16 kHz sampling rate
\item \textbf{Length normalization}: Pad or truncate to 64,600 samples (approximately 4 seconds)
\item \textbf{Silence trimming}: Remove leading/trailing silence using energy-based voice activity detection
\item \textbf{Normalization}: Apply peak normalization to [-1, 1] range
\end{enumerate}

For models using Wav2Vec2 features, we extract embeddings from the pre-trained \texttt{facebook/wav2vec2-base-960h} model with frozen CNN encoder layers.

\subsection{Data Augmentation and Normalization}
During training, we apply SpecAugment-style augmentation with frequency masking (parameter $F=10$) and time masking (parameter $T=10$) to improve generalization. Focal loss \cite{focalloss} is used to address the class imbalance (approximately 1:9 bonafide to spoof ratio in training).

\section{Train/Validation/Test Splits}

\subsection{Data Partitioning Strategy}
We follow the official ASVspoof 2019 LA partition:
\begin{itemize}
\item \textbf{Training}: Uses only known attack types (A01--A06) to simulate real-world scenarios where future attacks are unknown
\item \textbf{Development}: Used for hyperparameter tuning and early stopping
\item \textbf{Evaluation}: Contains 13 unseen attack types (A07--A19) to test generalization
\end{itemize}

This partition ensures that evaluation performance reflects the model's ability to generalize to novel, unseen spoofing methods---the core challenge in audio deepfake detection.

