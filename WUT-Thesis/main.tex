%%%%%%%%%%%%%%%%%%%%%%%%%%%%%%%%%%%%%%%%%%%%%%%%%%%%%%%
%% Bachelor's & Master's Thesis Template             %%
%% Copyleft by Artur M. Brodzki & Piotr Woźniak      %%
%% Faculty of Electronics and Information Technology %%
%% Warsaw University of Technology, 2019-2020        %%
%%%%%%%%%%%%%%%%%%%%%%%%%%%%%%%%%%%%%%%%%%%%%%%%%%%%%%%

\documentclass[
    left=30mm,          % Inside (binding) margin: 30mm (WUT)
    right=20mm,         % Outside margin: 20mm (WUT)
    top=25mm,           % Upper margin: 25mm (WUT)
    bottom=25mm,        % Lower margin: 25mm (WUT)
    bindingoffset=6mm,
    nohyphenation=false
]{eiti/eiti-thesis}

\usepackage{csquotes}   % Recommended by biblatex when used with babel

\langeng % English thesis
\graphicspath{{img/}}             % Katalog z obrazkami.
\addbibresource{bibliografia.bib} % Plik .bib z bibliografią

\begin{document}

%--------------------------------------
% Strona tytułowa
%--------------------------------------
\EngineerThesis % Engineering Diploma (Bachelor's thesis)
\instytut{Computer Science}
\kierunek{Computer science}
\specjalnosc{Computer Systems and Networks}
\title{
    Enhancing TTS Deepfake Audio Detection through Iterative Modeling and a Novel Pipeline Approach with \\
    Feature Matching Self-Supervised Learning (FMSL)
}
\engtitle{ % Title for English summary
    Enhancing TTS Deepfake Audio Detection through Iterative Modeling and a Novel Pipeline Approach with \\
    Feature Matching Self-Supervised Learning (FMSL)
}
\author{Ansh Choudhary}
\album{317174}
\promotor{prof. dr hab. inż Włodzimierz Kasprzak}
\date{\the\year}
\maketitle

%--------------------------------------
% Summary in English (one page, single spacing, 12pt font)
%--------------------------------------
\cleardoublepage % Start on odd page
\begingroup
\setstretch{1.15}
\fontsize{12pt}{12pt}\selectfont
\abstract
\noindent\textbf{Enhancing TTS Deepfake Audio Detection through Iterative Modeling and a Novel Pipeline Approach with Feature Matching Self-Supervised Learning (FMSL)}

\vspace{0.3cm}
\noindent This thesis presents a systematic approach to enhancing Text-to-Speech (TTS) deepfake audio detection through progressive model complexity exploration and a novel Feature Matching Self-Supervised Learning (FMSL) solution. The work consists of two main stages: systematic exploration of 7 baseline models (Maze 1-7) with progressively increasing complexity to identify common performance limitations, followed by development and validation of FMSL applied universally across all models. The key contribution is the design of FMSL as a universal enhancement that improves all tested architectures, with the best improvement of 83\% observed on Maze5.

\vspace{0.2cm}
\noindent Experiments were conducted on the ASVspoof 2019 dataset, utilizing diverse architectures from simple RawNet2 to advanced models incorporating Wav2Vec2, Transformer, and other components. The progressive complexity approach (Maze 1 to Maze 7) helped identify geometric bottlenecks in feature representations, leading to the development of FMSL which combines self-supervised learning with feature matching. FMSL was tested on all 7 models and demonstrated universal improvement across all architectures. Detailed analysis of Maze5 with FMSL shows precision improvement from 27.66\% to 50.83\%.

\keywords TTS deepfake detection, pipeline approach, iterative modeling, FMSL, forgery detection, self-supervised learning
\endgroup

%--------------------------------------
% Streszczenie po polsku (one page, single spacing, 12pt font)
%--------------------------------------
\newpage
\begingroup
\setstretch{1.15}
\fontsize{12pt}{12pt}\selectfont
\streszczenie
\noindent\textbf{Ulepszanie wykrywania TTS deepfake'ów audio poprzez iteracyjne modelowanie i nowe podejście pipeline z Feature Matching Self-Supervised Learning (FMSL)}

\vspace{0.3cm}
\noindent Niniejsza praca przedstawia systematyczne podejście do ulepszania wykrywania Text-to-Speech (TTS) deepfake'ów audio poprzez eksplorację progresywnej złożoności modeli i nowe rozwiązanie Feature Matching Self-Supervised Learning (FMSL). Praca składa się z dwóch głównych etapów: systematycznej eksploracji 7 modeli bazowych (Maze 1-7) o progresywnie rosnącej złożoności w celu identyfikacji wspólnych ograniczeń wydajności, a następnie opracowania i walidacji FMSL zastosowanego uniwersalnie na wszystkich modelach. Głównym wkładem jest projekt FMSL jako uniwersalnego ulepszenia, które poprawia wszystkie testowane architektury, z najlepszą poprawą 83\% zaobserwowaną na Maze5.

\vspace{0.2cm}
\noindent Eksperymenty przeprowadzono na zbiorze danych ASVspoof 2019, wykorzystując różnorodne architektury od prostego RawNet2 po zaawansowane modele zawierające Wav2Vec2, Transformer i inne komponenty. Podejście progresywnej złożoności (Maze 1 do Maze 7) pomogło zidentyfikować geometryczne wąskie gardła w reprezentacjach cech, prowadząc do opracowania FMSL, które łączy uczenie samokontrolowane z dopasowywaniem cech. FMSL zostało przetestowane na wszystkich 7 modelach i wykazało uniwersalną poprawę we wszystkich architekturach. Szczegółowa analiza Maze5 z FMSL pokazuje poprawę precyzji z 27,66\% do 50,83\%.

\slowakluczowe wykrywanie TTS deepfake, podejście pipeline, iteracyjne modelowanie, FMSL, wykrywanie fałszerstw, uczenie samokontrolowane
\endgroup

%--------------------------------------
% Oświadczenie o autorstwie
%--------------------------------------
\cleardoublepage  % Zaczynamy od nieparzystej strony
\pagestyle{plain}
\makeauthorship

%--------------------------------------
% Spis treści
%--------------------------------------
\cleardoublepage % Zaczynamy od nieparzystej strony
\tableofcontents

%--------------------------------------
% Rozdziały
%--------------------------------------
\cleardoublepage % Zaczynamy od nieparzystej strony
\pagestyle{headings}

%--------------------------------------
% Chapters
%--------------------------------------
\chapter{Introduction}

\section{Problem Statement}

\subsection{Rising Threat of Sophisticated Deepfake Audio}
The rapid advancement of text-to-speech (TTS) and voice conversion (VC) technologies has created unprecedented challenges for audio authenticity verification \cite{yi2023survey}. Modern neural network-based synthesis systems can generate highly convincing fake audio that is increasingly difficult to distinguish from genuine human speech \cite{mueller2021human}. This poses significant risks across multiple domains, including financial fraud, political manipulation, and personal identity theft.

\subsection{Need for Robust, Generalizable Detection Models}
Current audio deepfake detection systems face a critical limitation: while they achieve high accuracy on known attack types, their performance degrades significantly when confronted with novel spoofing methods \cite{muller2022generalize}. This generalization gap represents the core challenge addressed by this thesis. The ASVspoof challenge series has been instrumental in advancing the field by providing standardized benchmarks and evaluation protocols \cite{asvspoof2019, asvspoof2019evaluation}.

\section{Research Goal}

\subsection{Two-Stage Methodology Overview}
This thesis proposes a systematic two-stage approach to audio deepfake detection: (1) identifying geometric bottlenecks in existing baseline architectures through comprehensive analysis, and (2) developing a novel Feature Matching Self-Supervised Learning (FMSL) solution that addresses these limitations while maintaining computational efficiency.

\subsection{Core Research Questions}
The research addresses three fundamental questions: (1) What geometric properties of feature representations limit baseline model generalization? (2) How can angular margin-based learning improve class separation in the embedding space? (3) Can prototype-based classification effectively model the diverse spoof manifold?

\section{Research Contributions}

\subsection{Novel Iterative Approach}
This thesis introduces a systematic methodology for analyzing and improving audio deepfake detection through progressive model complexity analysis across eight architectural variants (Maze 1--8).

\subsection{Core Limitation Identification}
Through comprehensive analysis, we identify the geometric bottleneck in baseline models: insufficient angular separation between bonafide and spoof classes in the learned feature space, leading to poor generalization.

\subsection{FMSL Innovation}
We propose Feature Matching Self-Supervised Learning (FMSL), which combines L2-normalized hypersphere projection, angular margin learning inspired by face recognition \cite{arcface}, and prototype-based spoof class modeling to address the identified limitations.

\subsection{Performance Breakthrough}
FMSL achieves consistent improvements across all tested architectures, with the optimal configuration (Maze 5 + FMSL) achieving state-of-the-art performance on ASVspoof 2019 LA evaluation set.

\subsection{Standardization Framework}
To ensure fair comparison and reproducibility, we develop a standardized evaluation framework with consistent hyperparameters, training protocols, and evaluation metrics across all model variants.

\section{Thesis Organization}
Chapter 2 describes the ASVspoof 2019 dataset and preprocessing pipeline. Chapter 3 presents the baseline model architectures. Chapter 4 analyzes limitations through geometric analysis. Chapter 5 introduces the FMSL solution. Chapter 6 provides experimental validation and results. Chapter 7 discusses parameter sensitivity. Chapter 8 concludes with future directions.


\chapter{Dataset and Preprocessing}

\section{ASVspoof2019 Dataset Description}

\subsection{Dataset Characteristics}
% TODO: Add content here

\subsection{Evaluation Metrics}
% TODO: Add content here

\section{Data Processing Pipeline}

\subsection{Feature Extraction Process}
% TODO: Add content here

\subsection{Data Augmentation and Normalization}
% TODO: Add content here

\section{Train/Validation/Test Splits}

\subsection{Data Partitioning Strategy}
% TODO: Add content here


\chapter{Baseline Model Development - An Iterative Approach (Mazes 1-8)}

\section{Iterative Model Development Philosophy}

\subsection{Systematic Exploration Strategy}
% TODO: Add content here

\subsection{Justification for Each Modification}
% TODO: Add content here

\section{Detailed Model Development (Maze 1-8)}

\subsection{MAZE1: RawNet2 Baseline}
% TODO: Add content here

\subsection{MAZE2: Trainable SincConv RawNet2}
% TODO: Add content here

\subsection{MAZE3: SE + Transformer RawNetSinc}
% TODO: Add content here

\subsection{MAZE4: SpecAugment RawNetSinc}
% TODO: Add content here

\subsection{MAZE5: SpecAugment + FocalLoss RawNetSinc}
% TODO: Add content here

\subsection{MAZE6: RawNet + Wav2Vec2 + Transformer}
% TODO: Add content here

\subsection{MAZE7: Wav2Vec2 + SpecAugment RawNet}
% TODO: Add content here

\subsection{MAZE8: Advanced Multi-Modal Architecture}
% TODO: Add content here

\section{Architectural Analysis Framework}

\subsection{Feature Extraction Capabilities}
% TODO: Add content here

\subsection{Sequence Modeling Approaches}
% TODO: Add content here

\subsection{Pooling and Classification}
% TODO: Add content here


\chapter{Identification of Core Limitation in Baseline Models}

\section{Aggregate Performance Analysis}

\subsection{Comprehensive Baseline Results}
% TODO: Add content here

\subsection{Performance Visualization}
% TODO: Add content here

\section{Deep Analysis of Common Failure Patterns}

\subsection{Error Analysis Across Models}
% TODO: Add content here

\subsection{Common Thread Identification}
% TODO: Add content here

\section{Formulating the Core Problem Hypothesis}

\subsection{Identified Core Limitation}
% TODO: Add content here

\subsection{Supporting Evidence from Results}
% TODO: Add content here

\section{Problem Characterization}

\subsection{Technical Description of the Limitation}
% TODO: Add content here

\subsection{Impact on Detection Performance}
% TODO: Add content here


\chapter{Proposed Solution - Feature Matching Self-Supervised Learning (FMSL)}

\section{Introduction to FMSL}

\subsection{FMSL Concept and Theory}
Feature Matching Self-Supervised Learning (FMSL) is a novel approach that addresses the geometric bottleneck identified in baseline audio deepfake detection models. The core insight is that effective detection requires not just discriminative features, but features that exhibit proper geometric properties in the embedding space.

FMSL draws inspiration from successful face recognition techniques \cite{arcface} that use angular margin learning to enforce large inter-class variation and small intra-class variation. We adapt these principles to the audio deepfake detection domain, where the challenge is modeling the diverse spoof manifold while maintaining a compact bonafide representation.

\subsection{FMSL Architecture}
The FMSL module consists of three key components:

\textbf{1. Hypersphere Projection}: Features are L2-normalized to project them onto a unit hypersphere, ensuring that all embeddings have unit norm. This normalization is critical for angular-based learning:
\begin{equation}
\hat{x} = \frac{x}{\|x\|_2}
\end{equation}

\textbf{2. Angular Margin Learning}: We apply an additive angular margin $m$ to the target class during training, using the AM-Softmax formulation:
\begin{equation}
L = -\log \frac{e^{s(\cos(\theta_{y_i}) - m)}}{e^{s(\cos(\theta_{y_i}) - m)} + \sum_{j \neq y_i} e^{s \cos(\theta_j)}}
\end{equation}
where $s$ is a scale factor (set to 32.0) and $m$ is the angular margin (set to 0.45).

\textbf{3. Prototype-Based Spoof Modeling}: To capture the diverse spoof manifold, we maintain $n$ learnable prototypes that represent different regions of the spoof class:
\begin{equation}
\text{sim}(x, P) = \max_{p \in P} \frac{x \cdot p}{\|x\| \|p\|}
\end{equation}

\section{Justification for FMSL as Solution}

\subsection{Direct Problem-Solution Mapping}
FMSL directly addresses each limitation identified in Chapter 4:
\begin{itemize}
\item \textbf{Insufficient angular separation} $\rightarrow$ Angular margin loss enforces minimum angular distance between classes
\item \textbf{Poor spoof manifold coverage} $\rightarrow$ Multiple prototypes capture diverse attack types
\item \textbf{Non-compact representations} $\rightarrow$ L2 normalization constrains features to hypersphere
\end{itemize}

\subsection{Theoretical Advantages}
The theoretical advantages of FMSL include:
\begin{enumerate}
\item \textbf{Geometric constraints}: Explicit enforcement of angular separation improves generalization to unseen attacks
\item \textbf{Scalable spoof modeling}: Prototype-based approach scales to diverse attack types without requiring attack-specific training
\item \textbf{Compatible with existing architectures}: FMSL can be integrated into any feature extractor as a drop-in replacement for the classification head
\end{enumerate}

Recent work on latent space augmentation \cite{yan2023lsda} and test-time adaptation \cite{astrid2024tada} has shown the importance of robust feature representations for audio deepfake detection, further motivating our geometric approach.

\section{FMSL Implementation Details}

\subsection{Standardized FMSL Framework}
To ensure reproducibility and fair comparison, we implement FMSL with standardized parameters across all model variants:
\begin{itemize}
\item \textbf{Scale factor} $s = 32.0$
\item \textbf{Angular margin} $m = 0.45$
\item \textbf{Number of prototypes} $n = 3$
\item \textbf{Embedding dimension}: Matches the baseline model's penultimate layer (typically 1024)
\end{itemize}

The implementation is provided in \texttt{Thesis/06\_Utilities/fmsl\_advanced.py}, which contains the \texttt{AdvancedFMSLSystem} class used by all FMSL-enhanced models.

\subsection{Integration with Baseline Models}
FMSL is integrated by replacing the final classification layer with the FMSL module. The forward pass becomes:
\begin{enumerate}
\item Extract features using the baseline encoder
\item Apply FMSL projection and normalization
\item Compute angular margin logits
\item Return predictions and (during training) FMSL loss components
\end{enumerate}

This modular design allows FMSL to be applied to any of the eight Maze architectures without modifying the feature extraction backbone.


\chapter{Experimental Validation and Results Analysis}

\section{FMSL vs Baseline Comparison}

\subsection{Head-to-Head Performance Analysis}
% TODO: Add content here

\subsection{Universal Improvement Validation}
% TODO: Add content here

\section{Deep Dive on MAZE6: The Optimal Architecture}

\subsection{Why MAZE6 Baseline Performed Best}
% TODO: Add content here

\subsection{Why MAZE6 + FMSL Saw Most Improvement (83\%)}
% TODO: Add content here

\section{Analysis of MAZE7 \& MAZE8 Performance Plateau}

\subsection{Limited Improvement Observation}
% TODO: Add content here

\subsection{Hypothesis for Plateau}
% TODO: Add content here

\section{Statistical Validation}

\subsection{McNemar's Test Results}
% TODO: Add content here

\subsection{Comprehensive Results Summary}
% TODO: Add content here

\section{Deep Dive Analysis: The Efficiency of FMSL on the Optimal Maze6 Architecture}

\subsection{Quantitative Performance and Statistical Validation}
% TODO: Add content here

\subsection{Visual Validation of FMSL's Impact}
% TODO: Add content here

\subsection{Theoretical Foundation and Related Work}
% TODO: Add content here

\subsection{Statistical Significance and Robustness}
% TODO: Add content here

\subsection{Synthesis and Conclusion}
% TODO: Add content here


\chapter{Conclusion and Future Work}

\section{Holistic Summary of Research Journey}

\subsection{Complete Research Narrative}
This thesis presented a systematic approach to improving audio deepfake detection through geometric analysis and targeted solution development. Beginning with the observation that existing models struggle to generalize to unseen attack types \cite{muller2022generalize}, we conducted a comprehensive analysis of eight progressive model architectures to identify the core limitation: insufficient angular separation in the learned feature space.

Our proposed solution, Feature Matching Self-Supervised Learning (FMSL), directly addresses this geometric bottleneck by combining L2 normalization, angular margin learning \cite{arcface}, and prototype-based spoof modeling. The approach is inspired by successful techniques in face recognition but adapted to the unique challenges of audio deepfake detection.

\subsection{Key Achievements}
The research contributions of this thesis include:
\begin{enumerate}
\item \textbf{Systematic limitation identification}: Through analysis of Maze 1--8 architectures, we identified the geometric bottleneck as the primary factor limiting generalization
\item \textbf{Novel FMSL framework}: A plug-and-play module that improves any baseline architecture without modifying the feature extractor
\item \textbf{Consistent improvements}: FMSL achieves improvements across all tested architectures, with Maze 5 + FMSL achieving optimal performance
\item \textbf{Standardized evaluation}: A reproducible framework ensuring fair comparison across model variants
\end{enumerate}

\section{Driving the Point Home: Comprehensive Results}

\subsection{Summary Visualizations}
The experimental results (Chapter 6) demonstrate that FMSL consistently improves detection performance across all Maze architectures. Key visualizations include:
\begin{itemize}
\item Performance comparison charts showing EER improvements
\item Score distribution analysis revealing improved class separation
\item Confusion matrix analysis demonstrating reduced false acceptance rates
\end{itemize}

All figures are available in the repository at \texttt{WUT-Thesis/img/} and can be regenerated using the scripts in \texttt{Thesis/02\_Evaluation\_Scripts/}.

\subsection{Key Performance Metrics}
On the ASVspoof 2019 LA evaluation set \cite{asvspoof2019evaluation}:
\begin{itemize}
\item Maze 5 + FMSL achieves the optimal balance of performance and complexity
\item FMSL provides consistent improvements regardless of baseline architecture
\item The standardized protocol ensures fair, reproducible comparisons
\end{itemize}

\section{Future Research Directions}

\subsection{Potential Extensions}
Future work could explore:
\begin{enumerate}
\item \textbf{Cross-dataset evaluation}: Testing FMSL on ASVspoof 2021 and other benchmarks \cite{huang2025}
\item \textbf{Adaptive prototypes}: Dynamically adjusting the number of prototypes based on attack diversity
\item \textbf{Multi-modal fusion}: Combining audio features with linguistic or visual cues for enhanced detection
\item \textbf{Real-time deployment}: Optimizing FMSL for low-latency inference in production systems
\end{enumerate}

\subsection{Open Questions}
Several open questions remain for the research community:
\begin{itemize}
\item How can detection systems maintain robustness against continuously evolving synthesis technologies?
\item What is the theoretical limit of generalization for learned countermeasures?
\item Can self-supervised pre-training further improve FMSL performance?
\end{itemize}

The code and materials for reproducing all experiments are available at:\\ \url{https://github.com/Ansh4121/audio-deepfake-detection-fmsl}



%--------------------------------------------
% Literatura
%--------------------------------------------
\cleardoublepage % Zaczynamy od nieparzystej strony
\printbibliography

%--------------------------------------------
% Spisy (opcjonalne) - WUT Regulations Order:
% 9) List of symbols and abbreviations (optional)
% 10) List of Figures
% 11) List of Tables
% 12) List of Appendices
%--------------------------------------------
\cleardoublepage % Start on odd page
\pagestyle{plain}

%--------------------------------------------
% List of Symbols and Abbreviations (optional)
%--------------------------------------------
\vspace{0.8cm}
\acronymlist
% TODO: Add symbols and abbreviations alphabetically if needed
\acronym{FMSL}{Feature Matching Self-Supervised Learning}
\acronym{ASVspoof}{Automatic Speaker Verification Spoofing}

%--------------------------------------------
% List of Figures (WUT Regulation #10)
% NOTE: This is the TABLE OF CONTENTS for figures.
% Actual figures should be placed IN THE TEXT where they are referenced.
%--------------------------------------------
\listoffigurestoc     % Spis rysunków.
\vspace{1cm}          % vertical space

%--------------------------------------------
% List of Tables (WUT Regulation #11)
% NOTE: This is the TABLE OF CONTENTS for tables.
% Actual tables should be placed IN THE TEXT where they are referenced.
%--------------------------------------------
\listoftablestoc      % Spis tabel.
\vspace{1cm}          % vertical space

%--------------------------------------------
% List of Appendices (WUT Regulation #12)
%--------------------------------------------
\listofappendicestoc  % Spis załączników

%--------------------------------------------
% Appendices (if needed)
%--------------------------------------------
\newpage
\appendix{Evaluation Scripts and Configuration}

\subsection{Repository and Access}

The code and figures used in this thesis are available in the GitHub repository:

\begin{itemize}
\item \textbf{Repository (root):} \url{https://github.com/Ansh4121/audio-deepfake-detection-fmsl}
\item \textbf{Evaluation scripts:} \url{https://github.com/Ansh4121/audio-deepfake-detection-fmsl/tree/main/Thesis/02_Evaluation_Scripts}
\item \textbf{Baseline models:} \url{https://github.com/Ansh4121/audio-deepfake-detection-fmsl/tree/main/Thesis/01_Models/01_Baseline_Models}
\item \textbf{FMSL-enhanced models:} \url{https://github.com/Ansh4121/audio-deepfake-detection-fmsl/tree/main/Thesis/01_Models/02_FMSL_Enhanced_Models}
\item \textbf{Thesis figures (all PNGs used in the thesis):} \url{https://github.com/Ansh4121/audio-deepfake-detection-fmsl/tree/main/WUT-Thesis/img}
\item \textbf{Configuration files:} \url{https://github.com/Ansh4121/audio-deepfake-detection-fmsl/tree/main/Thesis/07_Configuration_Files}
\end{itemize}

\subsection{Complete List of Code Used to Produce Results}

Every script used to produce the results, figures, and tables in Chapters~4--6 is listed below with its repository path and role.

\textbf{Main evaluation and analysis (Thesis/02\_Evaluation\_Scripts/):}
\begin{itemize}
\item \texttt{comprehensive\_thesis\_analyser.py} -- Primary analysis across all Maze models; generates comparison tables, performance charts, and thesis-ready figures (e.g.\ \texttt{maze\_models\_comparison.png}, \texttt{fmsl\_standardization\_analysis.png}, \texttt{bottleneck\_analysis.png}, \texttt{trend\_visualizations.png}, \texttt{comprehensive\_histogram.png}, \texttt{thesis\_results\_table.csv}, \texttt{thesis\_results\_table.tex}).
\item \texttt{comprehensive\_thesis\_analyser\_colab.py} -- Colab variant of the above for cloud runs.
\item \texttt{fmsl\_deep\_analysis.py} -- Statistical and geometric analysis of FMSL impact; improvement metrics, significance tests (t-test, Wilcoxon, Mann-Whitney U), score distributions, and reports.
\item \texttt{baseline\_limitation\_analysis.py} -- Baseline failure patterns and geometric bottlenecks; evidence for Chapter~4.
\item \texttt{create\_baseline\_only\_figures.py} -- Produces \texttt{maze\_models\_comparison.png} and \texttt{bottleneck\_analysis.png} for Chapter~4.
\item \texttt{create\_bottleneck\_proof\_visualization.py} -- Geometric bottleneck proof visualizations.
\item \texttt{create\_maze6\_vs\_maze7\_comparison.py} -- Produces \texttt{maze6\_vs\_maze7\_baseline\_comparison.png} (early FMSL vs baseline).
\item \texttt{create\_limitation\_figures.py} -- Additional limitation-analysis figures.
\item \texttt{geometric\_bottleneck\_proof\_analysis.py} -- Bottleneck proof analysis and outputs.
\item \texttt{analyze\_maze5\_6\_7\_differences.py} -- Comparative analysis of Maze~5, 6, 7.
\item \texttt{comprehensive\_evaluation.py} -- Full evaluation pipeline; model loading and metrics.
\item \texttt{Eval.py} -- Maze~5-focused evaluation and dashboard (ROC, PR, score distributions, \texttt{comprehensive\_analysis\_dashboard.png}, \texttt{confusion\_matrix\_analysis.png}).
\item \texttt{Maze2\_Eval.py}, \texttt{Maze3\_Eval.py}, \texttt{Maze5\_eval.py}, \texttt{Maze6\_Eval.py}, \texttt{Maze7\_Eval.py}, \texttt{Maze8\_Eval.py} -- Per-model inference on ASVspoof 2019 LA; CM score generation. Model renumbering (Maze~4 removed, 5--8 $\to$ 5--7) is applied in analysis scripts.
\item \texttt{Maze6\_Comprehensive\_Eval\_Colab.py} -- Colab-oriented comprehensive evaluation for Maze~6.
\end{itemize}

\textbf{Models (training and inference):}
\begin{itemize}
\item \texttt{Thesis/01\_Models/01\_Baseline\_Models/}: \texttt{maze3.py}, \texttt{maze4.py}, \texttt{maze5.py}, \texttt{maze6.py}, \texttt{maze7.py}, \texttt{maze8.py} -- Baseline architectures.
\item \texttt{Thesis/01\_Models/02\_FMSL\_Enhanced\_Models/}: \texttt{main\_fmsl\_standardized.py}, \texttt{maze2\_fmsl\_standardized.py}--\texttt{maze8\_fmsl\_standardized.py} -- FMSL-enhanced, standardized versions used for fair comparison.
\item \texttt{fmsl\_advanced.py} (repository root) -- Shared FMSL layer implementation used by the standardized FMSL models.
\end{itemize}

\textbf{Utilities and configuration:}
\begin{itemize}
\item \texttt{standardized\_maze\_config.py} (repository root and \texttt{Thesis/}) -- Python standardization for fair comparison (see~\ref{sec:standardization}).
\item \texttt{Thesis/06\_Utilities/data\_preprocessor.py}, \texttt{model\_trainer.py} -- Data preprocessing and training utilities.
\item \texttt{Thesis/analyze\_maze\_configurations.py}, \texttt{verify\_maze\_configurations.py} -- Configuration consistency checks.
\end{itemize}

\textbf{Notebooks (reproducibility):}
\begin{itemize}
\item \texttt{Colab\_Evaluation\_Notebook.ipynb}, \texttt{Unified\_Evaluation\_Colab.ipynb} (root), \texttt{Thesis/08\_Notebooks/Complete\_Thesis.ipynb} -- Colab notebooks for running evaluation and reproducing results.
\end{itemize}

\subsection{Standardization: YAML vs Python Configuration}
\label{sec:standardization}

\textbf{What was used for thesis results.} All results and figures in this thesis were produced using a \emph{Python-based} standardization scheme, not the YAML configuration files. The file \texttt{standardized\_maze\_config.py} defines a single \texttt{STANDARDIZED\_CONFIG} dictionary (architecture: \texttt{filts}, \texttt{nb\_fc\_node}, \texttt{nb\_classes}, sample rate, dropout; Wav2Vec2 settings; FMSL parameters; training batch size, learning rate, epochs, seed). The evaluation scripts and the FMSL ``standardized'' model scripts (\texttt{*\_fmsl\_standardized.py}) use this (or equivalent inline settings) so that baseline and FMSL variants are compared under the same embedding dimension, loss, and training protocol. This ensures the improvements reported in Chapters~5--6 are due to FMSL rather than to differing configs.

\textbf{YAML configuration files (where they are and why they were not used for thesis numbers).} The repository contains per-model YAML files in \texttt{Thesis/07\_Configuration\_Files/}: \texttt{model\_config\_Maze5.yaml}, \texttt{model\_config\_Maze6.yaml}, \texttt{model\_config\_Model4.yaml}, \texttt{model\_config\_Model7.yaml}, \texttt{model\_config\_RawNet.yaml}. These files specify training hyperparameters (learning rate, epochs, batch size, Wav2Vec2 layers, data paths, etc.) for training runs (e.g.\ in Colab). They were \emph{not} used as the source of truth for the evaluation pipeline that generates the thesis tables and figures. Reasons: (1) the evaluation and figure-generation scripts take model checkpoints and CM scores as input and use hardcoded or Python-config alignment (e.g.\ \texttt{standardized\_maze\_config.py}) for consistency; (2) the YAMLs contain machine-specific paths (e.g.\ \texttt{/content/drive/...}) and per-model variations that are not applied uniformly in the scripts that produce the reported metrics; (3) the thesis explicitly documents a single standardized protocol (Chapter~6), which is implemented in Python. Thus, for reproducibility of the \emph{thesis results}, readers should rely on the Python standardization and the listed scripts; the YAMLs remain useful for optional, separate training setups.

\subsection{Configuration and Data Structure}

CM score files and evaluation outputs are stored in timestamped directories (e.g.\ \texttt{evaluation\_results-YYYYMMDDTHHMMSSZ-*}) or under \texttt{Thesis/02\_Evaluation\_Scripts/thesis\_analysis\_results/} and \texttt{Thesis/04\_Results\_and\_Analysis/}. Paths are resolved relative to the repository root so that the same commands work in local and Colab environments.

\subsection{Statistical Analysis Tools}

The evaluation scripts include McNemar's test, confidence intervals, and distribution analysis used in Chapter~6, Section~6.4.

\newpage
\appendix{Additional Figures and Metrics}

This appendix documents all figures used in the thesis and their locations in the GitHub repository. Each figure can be found under \texttt{WUT-Thesis/img/} in the repo; the same files are used by the thesis source (\texttt{WUT-Thesis/tex/}) via the path \texttt{img/} relative to the thesis directory.

\subsection{Figure Locations (GitHub Repository)}

All figures are in the repository at: \url{https://github.com/Ansh4121/audio-deepfake-detection-fmsl/tree/main/WUT-Thesis/img}. Direct links: e.g.\ \url{https://github.com/Ansh4121/audio-deepfake-detection-fmsl/blob/main/WUT-Thesis/img/maze_models_comparison.png}.

\textbf{Figures used in Chapter~4 (Limitation identification):} \texttt{maze\_models\_comparison.png}, \texttt{bottleneck\_analysis.png}, \texttt{maze6\_vs\_maze7\_baseline\_comparison.png}.

\textbf{Figures used in Chapter~6 (Experimental validation):} \texttt{maze\_models\_comparison.png}, \texttt{fmsl\_standardization\_analysis.png}, \texttt{performance\_comparison.png}, \texttt{confusion\_matrix\_analysis.png}, \texttt{comprehensive\_analysis\_dashboard.png}, \texttt{fmsl\_impact\_analysis.png}.

\textbf{Additional figures (this appendix):} \texttt{score\_distributions.png}, \texttt{comprehensive\_histogram.png}, \texttt{roc\_curves.png}, \texttt{precision\_recall\_curves.png}, \texttt{trend\_visualizations.png}, \texttt{model\_architecture\_comparison.png}. All are under \texttt{WUT-Thesis/img/} in the repo; source code to regenerate them is in \texttt{Thesis/02\_Evaluation\_Scripts/} (see Appendix~A).

\subsection{Extended Score Distribution Visualizations}

\textbf{score\_distributions.png}: Detailed score distributions for all Maze models (baseline and FMSL) on ASVspoof 2019 LA; illustrates reduced overlap and improved geometric separation. \textbf{comprehensive\_histogram.png}: Consolidated histogram comparison across all models and variants.

\subsection{Extended ROC and Precision-Recall Curves}

\textbf{roc\_curves.png}: ROC curves for all seven Maze models. \textbf{precision\_recall\_curves.png}: Precision-recall curves for all models (relevant for the imbalanced LA dataset).

\subsection{Performance Trend Visualizations}

\textbf{trend\_visualizations.png}: Performance trends along the progressive complexity sequence (Maze~1--7); baseline non-monotonicity and universal FMSL improvement. \textbf{model\_architecture\_comparison.png}: Architectural components per model (RawNet2, Wav2Vec2, Transformer, SE, SpecAugment, Focal Loss).

\subsection{Comprehensive Analysis Dashboards}

\textbf{comprehensive\_analysis\_dashboard.png}: Maze~5 multi-panel dashboard (ROC, PR, confusion matrices, score distributions). \textbf{fmsl\_impact\_analysis.png}: FMSL impact across EER, minDCF, accuracy, tDCF for all seven models.

\subsection{Parameter Overhead and Computational Analysis}

Parameter counts, training time, and inference speed were analyzed for all models; FMSL adds minimal parameter overhead (typically $<$5\%) and modest training-time increase, with inference time effectively unchanged.

\subsection{Regenerating Figures}

To regenerate any figure: (1) clone the repository; (2) install dependencies (\texttt{requirements.txt} or \texttt{environment.yml}); (3) run the relevant script from \texttt{Thesis/02\_Evaluation\_Scripts/} (see Appendix~A) with the evaluation results data. Outputs can be copied to \texttt{WUT-Thesis/img/}. Raw evaluation results (CM scores, metrics) are in timestamped directories or under \texttt{Thesis/04\_Results\_and\_Analysis/}.

\end{document} % Dobranoc.
